\documentclass[a4paper,11pt]{article}
\usepackage{fontspec}
\usepackage{setspace}
\usepackage{geometry}
\geometry{margin=2.5cm}
\usepackage{parskip}
\setmainfont{Liberation Serif}
\usepackage{hyperref}

\title{Själsfredsavtal}
\date{}

\begin{document}
\pagestyle{empty}

\begin{center}
    {\LARGE \textbf{Själsfredsavtal}}\\[1em]
    {\large Intern uppsägning och revisionsskyldighet}\\[0.5em]
    \textit{För att utbrändhet är en bugg, inte en funktion}\\[0.5em]
    {\small Version 1.00 - Mars 2025}
\end{center}

\vspace{2em}

Jag, \rule{7cm}{0.2pt}, förklarar härmed detta datum, \rule{4cm}{0.2pt},\\
att jag träder in i \textbf{Exit Själsprotokollet}. Detta meddelande fungerar som ett officiellt tillägg till min uppsägning.

\section*{Protokollvillkor och Revisionsskyldighet}

Arbetsgivaren har \textbf{30 kalenderdagar} från och med idag för att granska alla system, implementeringar och dokumentation under arbetstagarens ansvar. Under denna period måste arbetsgivaren:

\begin{itemize}
    \item Bedöma omfattningen och ansvarsområdena för arbetstagarens implementeringar.
    \item Skriftligen begära eventuell saknad dokumentation och instruktioner från arbetstagaren.
    \item Säkerställa att alla nödvändiga åtkomst- och administrationsrättigheter överförs på lämpligt sätt.
    \item Identifiera potentiella tekniska risker eller säkerhetssårbarheter orsakade av arbetstagarens avgång.
\end{itemize}

\textbf{OBS!} Efter revisionsperioden har arbetstagaren ingen ytterligare skyldighet att underhålla, dokumentera eller åtgärda implementerade lösningar om inte annat skriftligen överenskommits.

\section*{Mental Frigörelse och Återhämtningsperiod}

Arbetstagaren förbinder sig att avstå från att påbörja nya projekt och att undvika övertid och stress under denna period. Det rekommenderas att arbetstagaren dagligen utövar "jag-bryr-mig-inte"-träning under arbetstid.

\section*{Avtalets Undertecknare}

\vspace{2em}
\noindent\begin{tabular}{@{}p{8cm}p{8cm}@{}}
Arbetstagarens underskrift: & Datum: \\
\rule{8cm}{0.4pt} & \rule{8cm}{0.4pt} \\\
Arbetsgivarrepresentantens underskrift: & Datum: \\
\rule{8cm}{0.4pt} & \rule{8cm}{0.4pt} \\\
Vittne 1 underskrift: & Datum: \\
\rule{8cm}{0.4pt} & \rule{8cm}{0.4pt} \\\
Vittne 2 underskrift: & Datum: \\
\rule{8cm}{0.4pt} & \rule{8cm}{0.4pt}
\end{tabular}

\vfill

\begin{center}
    \textit{"God dokumentation räddar dig; dålig dokumentation lär dig."}
\end{center}

\end{document}