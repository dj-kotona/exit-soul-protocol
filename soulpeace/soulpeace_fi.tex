\documentclass[a4paper,11pt]{article}
\usepackage{fontspec}
\usepackage{setspace}
\usepackage{geometry}
\geometry{margin=2.5cm}
\usepackage{parskip}
\setmainfont{Liberation Serif}
\usepackage{hyperref}

\title{Sielunrauhasopimus}
\date{}

\begin{document}
\pagestyle{empty}

\begin{center}
    {\LARGE \textbf{Sielunrauhasopimus}}\\[1em]
    {\large Sisäinen irtisanoutuminen ja auditointivelvoite}\\[0.5em]
    \textit{Koska burnout ei ole ominaisuus, vaan bugi}\\[0.5em]
    {\small Versio 1.00 - Maaliskuu 2025}
\end{center}

\vspace{2em}

Minä, \rule{7cm}{0.2pt}, ilmoitan tänään, \rule{4cm}{0.2pt},\\
että olen siirtynyt \textbf{Sielunrauhaprotokollaan}. Tämä ilmoitus toimii virallisena lisäyksenä irtisanomisilmoitukseeni.

\section*{Protokollan ehdot ja auditointivelvoite}

Työnantajalla on tästä päivästä lähtien \textbf{30 kalenteripäivää} aikaa auditoida työntekijän vastuulle kuuluvat järjestelmät, toteutukset ja dokumentaatiot. Auditoinnin aikana työnantajan tulee:

\begin{itemize}
    \item Selvittää työntekijän toteutusten laajuus ja vastuut.
    \item Pyytää työntekijältä puuttuvat dokumentaatiot ja ohjeistukset kirjallisesti.
    \item Varmistaa, että kaikki tarvittavat pääsy- ja hallintaoikeudet on siirretty asianmukaisesti.
    \item Käydä läpi mahdolliset tekniset riskit tai turvallisuusaukot, joita työntekijän lähtö voi aiheuttaa.
\end{itemize}

\textbf{HUOM!} Auditointivelvoitteen jälkeen työntekijällä ei ole vastuuta eikä velvollisuutta ylläpitää, dokumentoida tai korjata toteutettuja ratkaisuja, ellei erillisellä sopimuksella toisin sovita.

\section*{Henkinen irtautuminen ja palautumisjakso}

Työntekijä sitoutuu olemaan aloittamatta uusia projekteja sekä välttämään ylitöitä ja stressiä tänä aikana. Suositellaan työntekijälle päivittäistä \textit{"ei voisi vähempää kiinnostaa"} -harjoittelua työpäivän aikana.

\section*{Sopimuksen osapuolet}

\vspace{2em}
\noindent\begin{tabular}{@{}p{8cm}p{8cm}@{}}
Työntekijän allekirjoitus: & Päiväys: \\
\rule{8cm}{0.4pt} & \rule{8cm}{0.4pt} \\\
Työnantajan edustajan allekirjoitus: & Päiväys: \\
\rule{8cm}{0.4pt} & \rule{8cm}{0.4pt} \\\
Todistaja 1:n allekirjoitus: & Päiväys: \\
\rule{8cm}{0.4pt} & \rule{8cm}{0.4pt} \\\
Todistaja 2:n allekirjoitus: & Päiväys: \\
\rule{8cm}{0.4pt} & \rule{8cm}{0.4pt}
\end{tabular}

\vfill

\begin{center}
    \textit{"Hyvä dokumentaatio pelastaa, huono dokumentaatio opettaa."}
\end{center}

\end{document}
