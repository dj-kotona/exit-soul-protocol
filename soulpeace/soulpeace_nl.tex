\documentclass[a4paper,11pt]{article}
\usepackage{fontspec}
\usepackage{setspace}
\usepackage{geometry}
\geometry{margin=2.5cm}
\usepackage{parskip}
\setmainfont{Liberation Serif}
\usepackage{hyperref}

\title{Zielenrust Overeenkomst}
\date{}

\begin{document}
\pagestyle{empty}

\begin{center}
    {\LARGE \textbf{Zielenrust Overeenkomst}}\\[1em]
    {\large Interne opzegging en auditverplichting}\\[0.5em]
    \textit{Omdat burn-out een bug is, geen functie}\\[0.5em]
    {\small Versie 1.00 - Maart 2025}
\end{center}

\vspace{2em}

Ik, \rule{7cm}{0.2pt}, verklaar op deze datum, \rule{4cm}{0.2pt},\\
dat ik het \textbf{Exit Ziel Protocol} in ga. Deze kennisgeving dient als een officiële aanvulling op mijn ontslag.

\section*{Protocolvoorwaarden en Auditverplichting}

De werkgever heeft \textbf{30 kalenderdagen} vanaf vandaag om alle systemen, implementaties en documentatie onder de verantwoordelijkheid van de werknemer te controleren. Tijdens deze periode moet de werkgever:

\begin{itemize}
    \item De reikwijdte en verantwoordelijkheden van de implementaties van de werknemer beoordelen.
    \item Ontbrekende documentatie en instructies schriftelijk opvragen bij de werknemer.
    \item Ervoor zorgen dat alle noodzakelijke toegangs- en beheersrechten op passende wijze worden overgedragen.
    \item Mogelijke technische risico's of beveiligingsproblemen identificeren die veroorzaakt worden door het vertrek van de werknemer.
\end{itemize}

\textbf{LET OP!} Na de auditverplichting periode heeft de werknemer geen verdere verantwoordelijkheid of verplichting om de geïmplementeerde oplossingen te onderhouden, documenteren of repareren, tenzij anders schriftelijk overeengekomen.

\section*{Mentale Ontkoppeling en Herstelperiode}

De werknemer verbindt zich ertoe om tijdens deze periode geen nieuwe projecten te starten en overwerk en stress te vermijden. Het wordt aanbevolen dat de werknemer dagelijks de "het kan me niet schelen" training uitvoert tijdens werkuren.

\section*{Ondertekenaars van de Overeenkomst}

\vspace{2em}
\noindent\begin{tabular}{@{}p{8cm}p{8cm}@{}}
Handtekening werknemer: & Datum: \\
\rule{8cm}{0.4pt} & \rule{8cm}{0.4pt} \\\
Handtekening vertegenwoordiger werkgever: & Datum: \\
\rule{8cm}{0.4pt} & \rule{8cm}{0.4pt} \\\
Handtekening getuige 1: & Datum: \\
\rule{8cm}{0.4pt} & \rule{8cm}{0.4pt} \\\
Handtekening getuige 2: & Datum: \\
\rule{8cm}{0.4pt} & \rule{8cm}{0.4pt}
\end{tabular}

\vfill

\begin{center}
    \textit{"Goede documentatie redt je; slechte documentatie leert je."}
\end{center}

\end{document}