\documentclass[a4paper,11pt]{article}
\usepackage{fontspec}
\usepackage{setspace}
\usepackage{geometry}
\geometry{margin=2.5cm}
\usepackage{parskip}
\setmainfont{Liberation Serif}
\usepackage{hyperref}

\title{Umowa Spokoju Duszy}
\date{}

\begin{document}
\pagestyle{empty}

\begin{center}
    {\LARGE \textbf{Umowa Spokoju Duszy}}\\[1em]
    {\large Wewnętrzna rezygnacja i obowiązek audytu}\\[0.5em]
    \textit{Ponieważ wypalenie zawodowe to błąd, a nie funkcja}\\[0.5em]
    {\small Wersja 1.00 - Marzec 2025}
\end{center}

\vspace{2em}

Ja, \rule{7cm}{0.2pt}, oświadczam w dniu dzisiejszym, \rule{4cm}{0.2pt},\\
że wchodzę w \textbf{Protokół Wyjścia Duszy}. Niniejsze zawiadomienie służy jako oficjalna poprawka do mojej rezygnacji.

\section*{Warunki Protokołu i Obowiązek Audytu}

Pracodawca ma \textbf{30 dni kalendarzowych} od dziś, aby przeprowadzić audyt wszystkich systemów, wdrożeń i dokumentacji będących w zakresie odpowiedzialności pracownika. W tym okresie pracodawca musi:

\begin{itemize}
    \item Ocenić zakres i odpowiedzialność związaną z wdrożeniami pracownika.
    \item Zażądać na piśmie od pracownika wszelkiej brakującej dokumentacji i instrukcji.
    \item Zapewnić, że wszystkie niezbędne prawa dostępu i zarządzania zostaną odpowiednio przeniesione.
    \item Zidentyfikować potencjalne ryzyka techniczne lub luki w zabezpieczeniach spowodowane odejściem pracownika.
\end{itemize}

\textbf{UWAGA!} Po okresie obowiązku audytu pracownik nie ponosi żadnej dalszej odpowiedzialności ani obowiązku utrzymywania, dokumentowania lub naprawiania wdrożonych rozwiązań, chyba że zostało to uzgodnione inaczej na piśmie.

\section*{Odłączenie Mentalne i Okres Regeneracji}

Pracownik zobowiązuje się powstrzymać od rozpoczynania nowych projektów oraz unikania nadgodzin i stresu w tym okresie. Zaleca się, aby pracownik codziennie praktykował podczas godzin pracy ćwiczenie "całkowicie mnie to nie obchodzi".

\section*{Sygnatariusze Umowy}

\vspace{2em}
\noindent\begin{tabular}{@{}p{8cm}p{8cm}@{}}
Podpis pracownika: & Data: \\
\rule{8cm}{0.4pt} & \rule{8cm}{0.4pt} \\\
Podpis przedstawiciela pracodawcy: & Data: \\
\rule{8cm}{0.4pt} & \rule{8cm}{0.4pt} \\\
Podpis świadka 1: & Data: \\
\rule{8cm}{0.4pt} & \rule{8cm}{0.4pt} \\\
Podpis świadka 2: & Data: \\
\rule{8cm}{0.4pt} & \rule{8cm}{0.4pt}
\end{tabular}

\vfill

\begin{center}
    \textit{"Dobra dokumentacja cię ratuje; zła dokumentacja cię uczy."}
\end{center}

\end{document}