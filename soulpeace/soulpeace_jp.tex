\documentclass[a4paper,11pt]{article}
\usepackage{fontspec}
\usepackage{setspace}
\usepackage{geometry}
\geometry{margin=2.5cm}
\usepackage{parskip}
\setmainfont{Liberation Serif}
\usepackage{hyperref}

\title{魂の平和協定}
\date{}

\begin{document}
\pagestyle{empty}

\begin{center}
    {\LARGE \textbf{魂の平和協定}}\\[1em]
    {\large 内的退職と監査義務}\\[0.5em]
    \textit{燃え尽き症候群はバグであり、機能ではないから}\\[0.5em]
    {\small バージョン 1.00 - 2025年3月}
\end{center}

\vspace{2em}

私、\rule{7cm}{0.2pt}は、本日\rule{4cm}{0.2pt}に、\\
\textbf{エグジット・ソウル・プロトコル}に入ることをここに宣言します。この通知は私の退職に対する公式な修正として機能します。

\section*{プロトコルの条件と監査義務}

雇用主は本日から\textbf{30暦日}以内に、従業員の責任下にあるすべてのシステム、実装、およびドキュメントを監査する必要があります。この期間中、雇用主は以下を行わなければなりません:

\begin{itemize}
    \item 従業員の実装の範囲と責任を評価すること。
    \item 不足しているドキュメントと指示を従業員に書面で要求すること。
    \item すべての必要なアクセス権と管理権が適切に移転されていることを確認すること。
    \item 従業員の退職によって引き起こされる可能性のある技術的リスクやセキュリティ上の脆弱性を特定すること。
\end{itemize}

\textbf{注意!}監査義務期間の後、従業員は書面で別途合意されない限り、実装されたソリューションを維持、文書化、または修正する更なる責任や義務を負いません。

\section*{精神的断絶と回復期間}

従業員はこの期間中、新しいプロジェクトを開始せず、残業やストレスを避けることを約束します。従業員が勤務時間中に毎日「全く気にしない」トレーニングを行うことをお勧めします。

\section*{協定署名者}

\vspace{2em}
\noindent\begin{tabular}{@{}p{8cm}p{8cm}@{}}
従業員署名: & 日付: \\
\rule{8cm}{0.4pt} & \rule{8cm}{0.4pt} \\\
雇用主代表署名: & 日付: \\
\rule{8cm}{0.4pt} & \rule{8cm}{0.4pt} \\\
証人1署名: & 日付: \\
\rule{8cm}{0.4pt} & \rule{8cm}{0.4pt} \\\
証人2署名: & 日付: \\
\rule{8cm}{0.4pt} & \rule{8cm}{0.4pt}
\end{tabular}

\vfill

\begin{center}
    \textit{"良いドキュメントはあなたを救い、悪いドキュメントはあなたに教えます。"}
\end{center}

\end{document}