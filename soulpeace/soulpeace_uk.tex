\documentclass[a4paper,11pt]{article}
\usepackage{fontspec}
\usepackage{setspace}
\usepackage{geometry}
\geometry{margin=2.5cm}
\usepackage{parskip}
\setmainfont{Liberation Serif}
\usepackage{hyperref}

\title{Угода про душевний спокій}
\date{}

\begin{document}
\pagestyle{empty}

\begin{center}
    {\LARGE \textbf{Угода про душевний спокій}}\\[1em]
    {\large Внутрішнє звільнення та зобов'язання аудиту}\\[0.5em]
    \textit{Тому що вигорання - це помилка, а не функція}\\[0.5em]
    {\small Версія 1.00 - Березень 2025}
\end{center}

\vspace{2em}

Я, \rule{7cm}{0.2pt}, оголошую сьогодні, \rule{4cm}{0.2pt},\\
що переходжу на \textbf{Протокол душевного спокою}. Це повідомлення є офіційним доповненням до мого звільнення.

\section*{Умови протоколу та зобов'язання з аудиту}

Роботодавець має \textbf{30 календарних днів} з цієї дати для аудиту всіх систем, впроваджень та документації, які знаходяться у відповідальності працівника. Протягом цього періоду роботодавець повинен:

\begin{itemize}
    \item Оцінити масштаби та відповідальність за впровадження працівника.
    \item Запитати від працівника письмово всю відсутню документацію та інструкції.
    \item Переконатися, що всі необхідні права доступу та управління передані належним чином.
    \item Виявити потенційні технічні ризики або прогалини в безпеці, які можуть виникнути після звільнення працівника.
\end{itemize}

\textbf{УВАГА!} Після періоду аудиту працівник більше не несе відповідальності і не зобов'язаний підтримувати, документувати або виправляти реалізовані рішення, якщо інше не передбачено окремою письмовою угодою.

\section*{Ментальне відсторонення та період відновлення}

Працівник зобов'язується утримуватись від початку нових проектів, уникати понаднормової роботи та стресу протягом цього періоду. Рекомендується щоденне тренування у форматі "мені байдуже" під час робочих годин.

\section*{Сторони угоди}

\vspace{2em}
\noindent\begin{tabular}{@{}p{8cm}p{8cm}@{}}
Підпис працівника: & Дата: \\
\rule{8cm}{0.4pt} & \rule{8cm}{0.4pt} \\\
Підпис представника роботодавця: & Дата: \\
\rule{8cm}{0.4pt} & \rule{8cm}{0.4pt} \\\
Підпис свідка 1: & Дата: \\
\rule{8cm}{0.4pt} & \rule{8cm}{0.4pt} \\\
Підпис свідка 2: & Дата: \\
\rule{8cm}{0.4pt} & \rule{8cm}{0.4pt}
\end{tabular}

\vfill

\begin{center}
    \textit{"Хороша документація рятує, погана — вчить."}
\end{center}

\end{document}
