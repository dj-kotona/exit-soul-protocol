\documentclass[a4paper,11pt]{article}
\usepackage{fontspec}
\usepackage{setspace}
\usepackage{geometry}
\geometry{margin=2.5cm}
\usepackage{parskip}
\setmainfont{Liberation Serif}
\usepackage{hyperref}

\title{Seelenfriedensabkommen}
\date{}

\begin{document}
\pagestyle{empty}

\begin{center}
    {\LARGE \textbf{Seelenfriedensabkommen}}\\[1em]
    {\large Innere Kündigung und Prüfungspflicht}\\[0.5em]
    \textit{Denn Burnout ist ein Fehler, keine Funktion}\\[0.5em]
    {\small Version 1.00 - März 2025}
\end{center}

\vspace{2em}

Ich, \rule{7cm}{0.2pt}, erkläre hiermit am \rule{4cm}{0.2pt},\\
dass ich in das \textbf{Exit-Soul-Protokoll} eintrete. Diese Mitteilung dient als offizielle Ergänzung zu meiner Kündigung.

\section*{Protokollbedingungen und Prüfungspflicht}

Der Arbeitgeber hat ab heute \textbf{30 Kalendertage} Zeit, um alle Systeme, Implementierungen und Dokumentationen unter der Verantwortung des Mitarbeiters zu prüfen. Während dieses Zeitraums muss der Arbeitgeber:

\begin{itemize}
    \item Den Umfang und die Verantwortlichkeiten der Implementierungen des Mitarbeiters bewerten.
    \item Fehlende Dokumentation und Anweisungen schriftlich vom Mitarbeiter anfordern.
    \item Sicherstellen, dass alle notwendigen Zugangs- und Verwaltungsrechte angemessen übertragen werden.
    \item Potenzielle technische Risiken oder Sicherheitslücken identifizieren, die durch das Ausscheiden des Mitarbeiters entstehen könnten.
\end{itemize}

\textbf{HINWEIS!} Nach Ablauf der Prüfungsfrist hat der Mitarbeiter keine weitere Verantwortung oder Verpflichtung, implementierte Lösungen zu warten, zu dokumentieren oder zu beheben, sofern nicht schriftlich anders vereinbart.

\section*{Geistige Loslösung und Erholungsphase}

Der Mitarbeiter verpflichtet sich, während dieser Zeit keine neuen Projekte zu beginnen und Überstunden sowie Stress zu vermeiden. Es wird empfohlen, dass der Mitarbeiter täglich das "Mir-ist-es-völlig-egal"-Training während der Arbeitszeit praktiziert.

\section*{Unterzeichner des Abkommens}

\vspace{2em}
\noindent\begin{tabular}{@{}p{8cm}p{8cm}@{}}
Unterschrift des Mitarbeiters: & Datum: \\
\rule{8cm}{0.4pt} & \rule{8cm}{0.4pt} \\\
Unterschrift des Arbeitgebervertreters: & Datum: \\
\rule{8cm}{0.4pt} & \rule{8cm}{0.4pt} \\\
Unterschrift des Zeugen 1: & Datum: \\
\rule{8cm}{0.4pt} & \rule{8cm}{0.4pt} \\\
Unterschrift des Zeugen 2: & Datum: \\
\rule{8cm}{0.4pt} & \rule{8cm}{0.4pt}
\end{tabular}

\vfill

\begin{center}
    \textit{"Gute Dokumentation rettet dich; schlechte Dokumentation lehrt dich."}
\end{center}

\end{document}