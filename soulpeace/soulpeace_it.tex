\documentclass[a4paper,11pt]{article}
\usepackage{fontspec}
\usepackage{setspace}
\usepackage{geometry}
\geometry{margin=2.5cm}
\usepackage{parskip}
\setmainfont{Liberation Serif}
\usepackage{hyperref}

\title{Accordo di Pace Interiore}
\date{}

\begin{document}
\pagestyle{empty}

\begin{center}
    {\LARGE \textbf{Accordo di Pace Interiore}}\\[1em]
    {\large Dimissioni interne e obbligo di verifica}\\[0.5em]
    \textit{Perché il burnout è un bug, non una funzionalità}\\[0.5em]
    {\small Versione 1.00 - Marzo 2025}
\end{center}

\vspace{2em}

Io, \rule{7cm}{0.2pt}, dichiaro in data odierna, \rule{4cm}{0.2pt},\\
di entrare nel \textbf{Protocollo di Uscita dell'Anima}. Questa notifica serve come emendamento ufficiale alle mie dimissioni.

\section*{Termini del Protocollo e Obbligo di Verifica}

Il datore di lavoro ha \textbf{30 giorni di calendario} da oggi per verificare tutti i sistemi, le implementazioni e la documentazione sotto la responsabilità del dipendente. Durante questo periodo, il datore di lavoro deve:

\begin{itemize}
    \item Valutare l'ambito e le responsabilità delle implementazioni del dipendente.
    \item Richiedere per iscritto al dipendente qualsiasi documentazione e istruzione mancante.
    \item Garantire che tutti i diritti di accesso e di gestione necessari siano trasferiti in modo appropriato.
    \item Identificare potenziali rischi tecnici o vulnerabilità di sicurezza causati dall'uscita del dipendente.
\end{itemize}

\textbf{ATTENZIONE!} Dopo il periodo di obbligo di verifica, il dipendente non ha alcuna ulteriore responsabilità o obbligo di mantenere, documentare o correggere le soluzioni implementate, salvo diverso accordo scritto.

\section*{Disimpegno Mentale e Periodo di Recupero}

Il dipendente si impegna ad astenersi dall'iniziare nuovi progetti e ad evitare straordinari e stress durante questo periodo. Si raccomanda al dipendente di praticare quotidianamente l'esercizio "non potrebbe importarmene di meno" durante l'orario di lavoro.

\section*{Firmatari dell'Accordo}

\vspace{2em}
\noindent\begin{tabular}{@{}p{8cm}p{8cm}@{}}
Firma del dipendente: & Data: \\
\rule{8cm}{0.4pt} & \rule{8cm}{0.4pt} \\\
Firma del rappresentante del datore di lavoro: & Data: \\
\rule{8cm}{0.4pt} & \rule{8cm}{0.4pt} \\\
Firma del testimone 1: & Data: \\
\rule{8cm}{0.4pt} & \rule{8cm}{0.4pt} \\\
Firma del testimone 2: & Data: \\
\rule{8cm}{0.4pt} & \rule{8cm}{0.4pt}
\end{tabular}

\vfill

\begin{center}
    \textit{"Una buona documentazione ti salva; una cattiva documentazione ti insegna."}
\end{center}

\end{document}